\documentclass[english,floatsintext,man]{apa6}

\usepackage{amssymb,amsmath}
\usepackage{ifxetex,ifluatex}
\usepackage{fixltx2e} % provides \textsubscript
\ifnum 0\ifxetex 1\fi\ifluatex 1\fi=0 % if pdftex
  \usepackage[T1]{fontenc}
  \usepackage[utf8]{inputenc}
\else % if luatex or xelatex
  \ifxetex
    \usepackage{mathspec}
    \usepackage{xltxtra,xunicode}
  \else
    \usepackage{fontspec}
  \fi
  \defaultfontfeatures{Mapping=tex-text,Scale=MatchLowercase}
  \newcommand{\euro}{€}
\fi
% use upquote if available, for straight quotes in verbatim environments
\IfFileExists{upquote.sty}{\usepackage{upquote}}{}
% use microtype if available
\IfFileExists{microtype.sty}{\usepackage{microtype}}{}

% Table formatting
\usepackage{longtable, booktabs}
\usepackage{lscape}
% \usepackage[counterclockwise]{rotating}   % Landscape page setup for large tables
\usepackage{multirow}		% Table styling
\usepackage{tabularx}		% Control Column width
\usepackage[flushleft]{threeparttable}	% Allows for three part tables with a specified notes section
\usepackage{threeparttablex}            % Lets threeparttable work with longtable

% Create new environments so endfloat can handle them
% \newenvironment{ltable}
%   {\begin{landscape}\begin{center}\begin{threeparttable}}
%   {\end{threeparttable}\end{center}\end{landscape}}

\newenvironment{lltable}
  {\begin{landscape}\begin{center}\begin{ThreePartTable}}
  {\end{ThreePartTable}\end{center}\end{landscape}}




% The following enables adjusting longtable caption width to table width
% Solution found at http://golatex.de/longtable-mit-caption-so-breit-wie-die-tabelle-t15767.html
\makeatletter
\newcommand\LastLTentrywidth{1em}
\newlength\longtablewidth
\setlength{\longtablewidth}{1in}
\newcommand\getlongtablewidth{%
 \begingroup
  \ifcsname LT@\roman{LT@tables}\endcsname
  \global\longtablewidth=0pt
  \renewcommand\LT@entry[2]{\global\advance\longtablewidth by ##2\relax\gdef\LastLTentrywidth{##2}}%
  \@nameuse{LT@\roman{LT@tables}}%
  \fi
\endgroup}


  \usepackage{graphicx}
  \makeatletter
  \def\maxwidth{\ifdim\Gin@nat@width>\linewidth\linewidth\else\Gin@nat@width\fi}
  \def\maxheight{\ifdim\Gin@nat@height>\textheight\textheight\else\Gin@nat@height\fi}
  \makeatother
  % Scale images if necessary, so that they will not overflow the page
  % margins by default, and it is still possible to overwrite the defaults
  % using explicit options in \includegraphics[width, height, ...]{}
  \setkeys{Gin}{width=\maxwidth,height=\maxheight,keepaspectratio}
\ifxetex
  \usepackage[setpagesize=false, % page size defined by xetex
              unicode=false, % unicode breaks when used with xetex
              xetex]{hyperref}
\else
  \usepackage[unicode=true]{hyperref}
\fi
\hypersetup{breaklinks=true,
            pdfauthor={},
            pdftitle={Still suspicious: The suspicious coincidence effect revisited},
            colorlinks=true,
            citecolor=blue,
            urlcolor=blue,
            linkcolor=black,
            pdfborder={0 0 0}}
\urlstyle{same}  % don't use monospace font for urls

\setlength{\parindent}{0pt}
%\setlength{\parskip}{0pt plus 0pt minus 0pt}

\setlength{\emergencystretch}{3em}  % prevent overfull lines

\ifxetex
  \usepackage{polyglossia}
  \setmainlanguage{}
\else
  \usepackage[english]{babel}
\fi

% Manuscript styling
\captionsetup{font=singlespacing,justification=justified}
\usepackage{csquotes}
\usepackage{upgreek}



\usepackage{tikz} % Variable definition to generate author note

% fix for \tightlist problem in pandoc 1.14
\providecommand{\tightlist}{%
  \setlength{\itemsep}{0pt}\setlength{\parskip}{0pt}}

% Essential manuscript parts
  \title{Still suspicious: The suspicious coincidence effect revisited}

  \shorttitle{The suspicious coincidence effect revisited}


  \author{Molly L. Lewis\textsuperscript{1,2}~\& Michael C. Frank\textsuperscript{3}}

  \def\affdep{{"", ""}}%
  \def\affcity{{"", ""}}%

  \affiliation{
    \vspace{0.5cm}
          \textsuperscript{1} Computation Institute, University of Chicago\\
          \textsuperscript{2} Department of Psychology, University of Wisconsin, Madison\\
          \textsuperscript{3} Department of Psychology, Stanford University  }

 % If no author_note is defined give only author information if available
      \newcounter{author}
                              \authornote{
            Correspondence concerning this article should be addressed to Molly L. Lewis. E-mail: \href{mailto:mollylewis@uchicago.edu}{\nolinkurl{mollylewis@uchicago.edu}}
          }
                                  
  \note{\(^*\)To whom correspondence should be addressed. Email:
\href{mailto:mollylewis@uchicago.edu}{\nolinkurl{mollylewis@uchicago.edu}}}

  \abstract{Enter abstract here. Each new line herein must be indented, like this
line.}
  \keywords{word learning, Bayesian inference, meta-analysis, concepts \\

    \indent Word count: X
  }




  \usepackage{setspace}
  \usepackage{float}
  \usepackage{graphicx}
  \AtBeginEnvironment{tabular}{\singlespacing}
  \usepackage{pbox}

\usepackage{amsthm}
\newtheorem{theorem}{Theorem}
\newtheorem{lemma}{Lemma}
\theoremstyle{definition}
\newtheorem{definition}{Definition}
\newtheorem{corollary}{Corollary}
\newtheorem{proposition}{Proposition}
\theoremstyle{definition}
\newtheorem{example}{Example}
\theoremstyle{remark}
\newtheorem*{remark}{Remark}
\begin{document}

\maketitle

\setcounter{secnumdepth}{0}



\subsection{Intro}\label{intro}

What is the suspicious coincidence effect?

(Spencer, Perone, Smith, \& Samuelson, 2011; F. Xu \& Tenenbaum, 2007;
Fei Xu \& Tenenbaum, 2007)

Why is it important?

Spencer et al. paper

Methodological differences:

\begin{itemize}
\tightlist
\item
  simultaneous vs.~sequential
\item
  3-1 vs.~1-3
\item
  blocking
\item
  same label vs.~different label
\end{itemize}

other evidence relevant on this replicatione

Our current paper reports 12 experiments, 10 of which were
pre-registered. We recover the suspicious coincidence effect with a
large effect size in both sequential and simultaneous presentation
conditions. The effect only occurs, however, in experiments where the
trial with one exemplar is presented \emph{before} the key trial with
three subordinate-consistent exemplars (the \enquote{suspicious
coincidence}). We attribute this difference to participants' awareness
of the possibility of subordinate generalizations following the
three-exemplar trial; in these conditions, we see a high level of
subordinate generalizations even for the one-exemplar trial (leading to
the absence of a difference between conditions). In sum, and contra
SPSS, the \enquote{suspicious coincidence} effect is robust to
sequential presentation. The effect is sensitive to some features of the
general experimental context, however, suggesting a potential
interpretation in terms of the pragmatics of the task.

\section{Methods}\label{methods}

We report how we determined our sample size, all manipulations, and all
measures in the study. All stimuli, experimental code, sample sizes, and
analyses were pre-registered (\url{https://osf.io/yekhj/}), with the
exception of Exps. 4 and 8.

\subsection{Participants}\label{participants}

Fifty participants were recruited on Amazon Mechanical Turk for each of
our 12 experiments (N = 600), and paid 40-50 cents for their
participation. Across all 12 experiments, 13\% of participants completed
more than one experiment. We report data from all participants in the
Main Text, but the pattern of reported findings holds when these
participants are excluded (see SI).

We determined our sample size on the basis of a pre-registered power
calculation using a meta-analytic estimate of the effect size from
studies conducted by XT and SPSS. The chosen sample size was
approximately twice the estimated sample size necessary to obtain a
power of 1.

\begin{figure}[t!]
 
 {\centering \includegraphics[width=0.5\linewidth]{figs/stim} 
 
 }
 
 \caption{Sample stimuli. Three superodinate (top), basic (middle), and subordinate (bottom) exemplars from the vegetable category.}\label{fig:unnamed-chunk-1}
 \end{figure}

\subsection{Stimuli}\label{stimuli}

Our stimuli closely replicated that of XT and SPSS. The linguistic
stimuli were 12 one-syllable novel labels (e.g., \enquote{wug}), and the
referent objects were three sets of 15 pictures from different basic
level categories (vegetables, vehicles and animals). Within each
category, five were subordinate exemplars (e.g.~green pepper), four were
basic level exemplars (e.g.~peppers), and six were superordinate
exemplars (e.g.~vegetables; Fig.~1). The exemplars were divided into a
learning and generalization set. For each category, the learning set
consisted of 3 subordinate, 2 basic, and 2 superordinate pictures
presented in different combinations on different trials (see Procedure).
The generalization set for each category consisted of the remaining 8
pictures. The learning and generalization sets were the same for all
participants.

\subsection{Procedure}\label{procedure}

\begin{table}

\caption{\label{tab:unnamed-chunk-2}Summary of our 12 experiments.}
\centering
\fontsize{12}{14}\selectfont
\begin{tabular}[t]{crrrrrrr}
\toprule
\multicolumn{2}{c}{ } & \multicolumn{4}{c}{Manipulations} & \multicolumn{1}{c}{ } \\
\cmidrule(l{2pt}r{2pt}){3-6}
Exp. & N & Timing & Order & Blocking & Label & Effect Size & Original 
Exp.\\
\midrule
1 & 50 & simult. & 1-3 & pseudo-random & same & 1.32 [1.24, 1.4] & XT E1/E2\\
2 & 50 & simult. & 1-3 & pseudo-random & same & 1.14 [1.06, 1.22] & XT E1/E2\\
3 & 50 & simult. & 1-3 & pseudo-random & diff. & 1.16 [1.08, 1.24] & \\
4 & 50 & seq. & 1-3 & pseudo-random & same & 1.42 [1.32, 1.52] & \\
5 & 50 & seq. & 1-3 & pseudo-random & diff. & 1.26 [1.18, 1.34] & \\
\addlinespace
6 & 50 & seq. & 1-3 & blocked & diff. & 1.31 [1.23, 1.39] & \\
7 & 50 & simult. & 3-1 & blocked & diff. & 0.02 [-0.06, 0.1] & SPSS ES1/ES2\\
8 & 50 & simult. & 3-1 & blocked & diff. & -0.06 [-0.14, 0.02] & \\
9 & 50 & simult. & 3-1 & blocked & same & -0.14 [-0.22, -0.06] & \\
10 & 50 & seq. & 3-1 & blocked & diff. & -0.44 [-0.52, -0.36] & SPSS E2/E3\\
\addlinespace
11 & 50 & seq. & 3-1 & pseudo-random & same & -0.31 [-0.39, -0.23] & \\
12 & 50 & seq. & 3-1 & blocked & same & -0.17 [-0.25, -0.09] & \\
\bottomrule
\multicolumn{8}{l}{\textsuperscript{1} N = sample size; Timing = presentation timing (sequential or simultaneous); Order =}\\
\multicolumn{8}{l}{relative ordering of 1 and 3 subordinate trials; Blocking = trials blocked by category or}\\
\multicolumn{8}{l}{pseudo-random; Label = same or different label in 1 and 3 trials; Effect size = Cohen's d}\\
\multicolumn{8}{l}{[95\% CI]; Original Exp. = corresponding experiment from prior literature.}\\
\end{tabular}
\end{table}

Participants were first introduced to a picture of a character
(\enquote{Mr.~Frog}) and instructions describing the task. They were
told that the character speaks a different language and their job was to
help the character find the toys he wants. Participants then advanced to
the main experiment, which consisted of a series of 12 trials on
separate screens. On each trial, one or three learning exemplars from
one of the three stimulus categories appeared at the top of the screen,
along with the following instructions: \enquote{Here {[}is a wug/are
three wugs{]}. Can you give Mr.~Frog all of the other wugs?.} Below the
learning exemplars, 24 generalization exemplars (8 from each of the 3
categories) were displayed in a 4\emph{x}6 grid. The order of
generalizaiton pictures was randomized across trials. Participants were
instructed to select the other category memebers (\enquote{To give a
wug, click on it below. When you have given all the wugs, click the Next
button.}). When an exemplar was selected, a red box appeared around the
picture, and participants were allowed to change their selections by
clicking on the picture a second time. The learning exemplars remained
visible at the top of the screen during the generalization task. Once
they had made their selections, participants advanced to the next trial
by clicking the \enquote{Next} button.

There were four trial types distinguished by the number and semantic
level of the learning exemplars: one subordinate exemplar, three
subordinate exemplars, three basic exemplars, and three superordate
exemplars. Each participant completed each trial type for each of the
three stimulus categories (vegetables, vehicles and animals).

Across 12 experiments, we manipulated four aspects of the trial design
that differed between the XT and SPSS studies (summarized in Table 1):
Presentation timing (simultaneous vs.~sequential), trial order (1-3
vs.~3-1), label (same vs.~different), and blocking (blocked
vs.~pseudo-random)\footnote{All experiments can be viewed directly at XXX.}.
We describe each of these factors in more detail below.

\subsubsection{Presentation Timing}\label{presentation-timing}

Presentation timing was the key, theoretically motivated experimental
design difference between the XT (E1 and
E2\footnote{XT E1 and E2 differed in the age of participants (adults vs. children), but we collapse across this difference for the present analyses.})
and SPSS (E2 and E3). In XT, the learning exemplars were presented
statically and simultaneously, while in SPSS, participants saw a
sequence of individual exemplars with each exemplar visible only for 1s
at a time. In the sequential design, three exemplar learning trials
displayed pictures at three different locations (left, middle, and
right) in a sequence that repeated twice, for a total of 6s.

We reproduced these design aspects in the simultaneous and sequential
versions of our experiments. In the single exemplar, sequential trials,
the exemplar appeared (1s) and disappeared (1s) for three repetitions.
The generalization pictures did not appear in the sequnetial condition
until after the training pictures has appeared for 6 seconds, but
remained visibile as participants selected generalization exemplars.

\subsubsection{Trial order}\label{trial-order}

In XT, the three one-subordinate trials occured first followed by all
other trial types. In contrast, in SPSS (E2 and E3), the
three-subordinate trials occured first. SPSS's replication of XT's
simultanous design (SPSS E1) used the 1-3 ordering.{[}This isn't
actually quite true: \enquote{The first block of trials always involved
either one exemplar or three subordinate-level exemplars from each
domain. The remaining blocks of trials were randomly ordered for each
participant}\ldots{} so 1 exemplar trials were first only half the
time?{]}.

\subsubsection{Labels}\label{labels}

XT used the same label for each category for the three-subordinate and
one-suborinate trials. SPSS used a different novel label on each of the
12 trials, such that the three-subordinate and one-subordinate trials
were refered to with distinct labels. Labels were randomized across
trials.

\subsubsection{Blocking}\label{blocking}

Finally, the studies by XT and SPSS differ in terms of whether the
trials were blocked by trial type: In XT, the first three trials were a
block of one-subordinate trials and the remaining 9 trials were
randomized, whereas SPSS blocked all four trial types. Within each
block, category order was randomized.

\subsection{Data analysis}\label{data-analysis}

The key prediction of the suspicious coincidence effect is that
participants should generalize to the basic level more often in
one-subordinate trials relative to three-subordinate trials. To measure
this, for each trial, we calculated the proportion generalizations to
subordinate exemplars within the same category (out of 2) and basic
exemplars within the same category (out of 2), and averaged across
categories for each participant. We estimated the difference between the
one-subordinate and three-subordinate conditions by calculating an
effect size (Cohen's \emph{d}) for each experiment. We then estimated
the influence of each our design manipulations on the overall effect
size by fitting a random-effect meta-analytic model with each of our
four manipulations as fixed effects. We used the metafor package
(Viechtbauer, 2010) in R to fit our meta-analytic models.

\section{Results}\label{results}

Figure 2 shows the mean proportion generalizations to the basic level in
the one- and three-subordinate trials for all 12 experiments, and Figure
3 shows the corresponding effect sizes (with XT and SPSS experiments
included for reference).

\begin{figure}
\centering
\includegraphics{xtmem_files/figure-latex/unnamed-chunk-3-1.pdf}
\caption{\label{fig:unnamed-chunk-3}Mean proportion generalizations to basic
level exemplars in the one (pink) and three (green) subordinate exemplar
conditions for all 12 of our experiments. Each facet corresponds to a
pairing of presentation timing (simultaneous vs.~sequential) and trial
order (1-3 vs.~3-1). Error bars are bootstrapped 95\% confidence
intervals.}
\end{figure}

In two exact replications of the XT method (XT E1 and X2), we replicate
the suspicious coincidence effect (Exp. 1: \emph{d} = 1.32 {[}1.24,
1.4{]}; Exp. 2: \emph{d} = 1.14 {[}1.06, 1.22{]}), with a magnitude
comparable to the original XT experiments (\emph{d} = 2 {[}1.73, 2.27{]}
and d = 1.01 {[}0.89, 1.13{]}). We also replicate the reversal in the
suspicious coincidence effect observed by SPSS (SPSS E2 and E3) in an
exact replication of their method (Exp. 10; \emph{d} = -0.44 {[}-0.52,
-0.36{]}), and with a magnitude comparable to the original experiments
(SPSS E2: \emph{d} = -0.61 {[}-0.81, -0.41{]}; SPSS E3: \emph{d} = -0.3
{[}-0.52, -0.08{]}).

Critically, however, the meta-analyic model across all 12 experiments
suggests that only trial order is a reliable predictor of the magnitude
of the susicpious coincidence effect. THE MODEL.

\begin{figure}
\centering
\includegraphics{xtmem_files/figure-latex/unnamed-chunk-4-1.pdf}
\caption{\label{fig:unnamed-chunk-4}Effect sizes for all 19 studies
conducted on the suspicious coincidence effect by XT (Xu \& Tenenbaum,
2007a), SPSS (Spencer, et al, 2011), and the current authors. The
top-bottoom facets indicate whether the single exemplar trial occurred
first (1-3) or second (3-1). The left-right facets indicate whether the
exemplars were presented simulateously as in XT or sequentially as in
SPSS. Point color indicates whether the single exemplar and three
subordinate exemplars received the same (grey) or different (black)
label. Point shape indicates whether trials were blocked by category
(circle) or pseudo-random (triangle). Points are jittered along the
x-axis for visibility. The red line reflects the meta-analytic estimate
of the effect size (for the XT experiments, standard deviations on
effect sizes are estimated from the SPSS replication). All error bars
are 95\% confidence intervals.}
\end{figure}

\begin{table}

\caption{\label{tab:unnamed-chunk-5}Meta-analytic model with manipulations as fixed effects.}
\centering
\fontsize{12}{14}\selectfont
\begin{tabular}[t]{lrrr}
\toprule
Fixed effect & beta & z-value & p-value\\
\midrule
Intercept & 1.37 [1.09, 1.65] & 9.48 & <.0001\\
Simultaneous vs. sequential timing & -0.13 [-0.33, 0.08] & -1.18 & 0.24\\
1-3 vs. 3-1 condition order & -1.48 [-1.77, -1.18] & -9.90 & <.0001\\
Different vs. same label & 0.03 [-0.21, 0.26] & 0.20 & 0.84\\
Blocked vs. pseudo-random trial order & -0.09 [-0.41, 0.24] & -0.52 & 0.6\\
\bottomrule
\end{tabular}
\end{table}

\section{Discussion}\label{discussion}

\newpage

\section{References}\label{references}

\setlength{\parindent}{-0.5in} \setlength{\leftskip}{0.5in}

\hypertarget{refs}{}
\hypertarget{ref-spencer2011learning}{}
Spencer, J. P., Perone, S., Smith, L. B., \& Samuelson, L. K. (2011).
Learning words in space and time: Probing the mechanisms behind the
suspicious-coincidence effect. \emph{Psychological Science},
\emph{22}(8), 1049--1057.

\hypertarget{ref-R-metafor}{}
Viechtbauer, W. (2010). Conducting meta-analyses in R with the metafor
package. \emph{Journal of Statistical Software}, \emph{36}(3), 1--48.
Retrieved from \url{http://www.jstatsoft.org/v36/i03/}

\hypertarget{ref-xu2007word}{}
Xu, F., \& Tenenbaum, J. (2007). Word learning as Bayesian inference.
\emph{Psychological Review}, \emph{114}(2), 245.

\hypertarget{ref-xu2007}{}
Xu, F., \& Tenenbaum, J. B. (2007). Sensitivity to sampling in Bayesian
word learning. \emph{Developmental Science}, \emph{10}(3), 288--297.






\end{document}
